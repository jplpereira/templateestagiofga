\chapter[Descrição da Instituição]{Descrição da Instituição}
%\addcontentsline{toc}{chapter}{Descrição da Instituição}

%\section{Descrição da Instituição}

O Tribunal de Contas da União (TCU) foi criado no dia 7 de novembro de 1890, por iniciativa do então 
Ministro da Fazenda, Rui Barbosa, através do Decreto nº 966-A, sendo norteado pelos princípios da 
autonomia, fiscalização, julgamento, vigilância e energia. Sua sede funcionou em diversos locais da 
cidade do Rio de Janeiro até 10 de janeiro de 1965, data da inauguração do Palácio do Tribunal de 
Contas na Esplanada dos Ministérios em Brasília.

O TCU é o órgão de controle externo do governo federal e auxilia o Congresso Nacional na missão de 
acompanhar a execução orçamentária e financeira do país e contribuir com o aperfeiçoamento da 
Administração Pública em benefício da sociedade. Para isso, tem como meta ser referência na 
promoção de uma Administração Pública efetiva, ética, ágil e responsável, através da 
fiscalização contábil, financeira, orçamentária, operacional e patrimonial dos órgãos e 
entidades públicas do país quanto à legalidade, legitimidade e economicidade.

Ele é órgão colegiado composto de nove ministros, sendo seis deles indicados pelo Congresso 
Nacional, um deles pelo presidente da República e os outros dois escolhidos entre auditores e 
membros do Ministério Público que funciona junto ao Tribunal. Suas deliberações são tomadas, em regra, 
pelo Plenário ou, nas hipóteses cabíveis, por uma das duas Câmaras.

Suas competências estão elencadas, principalmente, no Art. 71 da Constituição Federal de 1988, além 
de outros artigos da própria Constituição e de leis específicas. Sejam elas:

\begin{itemize}
    \item Apreciar as contas anuais do presidente da República;
    \item Julgar as contas dos administradores e demais responsáveis por dinheiros, bens e 
    valores públicos;
    \item Apreciar a legalidade dos atos de admissão de pessoal e de concessão de aposentadorias, 
    reformas e pensões civis e militares;
    \item Realizar inspeções e auditorias por iniciativa própria ou por solicitação do Congresso 
    Nacional;
    \item Fiscalizar as contas nacionais das empresas supranacionais;
    \item Fiscalizar a aplicação de recursos da União repassados a estados, ao Distrito Federal 
    e a municípios;
    \item Prestar informações ao Congresso Nacional sobre fiscalizações realizadas;
    \item Aplicar sanções e determinar a correção de ilegalidades e irregularidades em atos e 
    contratos;
    \item Sustar, se não atendido, a execução de ato impugnado, comunicando a decisão à Câmara 
    dos Deputados e ao Senado Federal;
    \item Emitir pronunciamento conclusivo, por solicitação da Comissão Mista Permanente de 
    Senadores e Deputados, sobre despesas realizadas sem autorização;
    \item Apurar denúncias apresentadas por qualquer cidadão, partido político, associação ou 
    sindicato sobre irregularidades ou ilegalidades na aplicação de recursos federais;
    \item Fixar os coeficientes dos fundos de participação dos estados, do Distrito Federal e 
    dos municípios e fiscalizar a entrega dos recursos aos governos estaduais e às prefeituras 
    municipais.
\end{itemize}

Em sua estrutura organizacional, o TCU possui as Secretarias de Controle Externo (Secex), que 
assessoram os ministros-relatores no controle de gestão e oferecem subsídios técnicos para o 
julgamento das contas e apreciação dos demais processos relativos às unidades jurisdicionadas ao 
TCU. 

Uma dessas secretarias é a Secretaria de Controle Externo da Educação (SecexEduc), que tem o 
objetivo de prestar suporte na realização de tarefas vinculadas às fiscalizações a cargo do TCU e 
nas ações e atividades inerentes ao exame de processos de contas ou relativos à apreciação de atos 
de gestão das unidades responsáveis pelos programas federais de governo da área de educação e 
aquelas que também recebem recursos federais para este fim.

Para prestar auxílio nas ações realizadas por essa secretaria e municiar os auditores responsáveis 
com informações para a tomada de decisões, foi criado o Núcleo de Análise de Dados e de Tratamento 
de Informações dentro da estrutura da SecexEduc.




